\footnotesize
\fancyhead[C]{\itГлава 10}
\linespread{1}
\normalsize
\par\text{Д\thispace\thinspaceо\thispace\thinspaceк\thispace\thinspaceа\thispace\thinspaceз\thispace\thinspaceа\thispace\thinspaceт\thispace\thinspaceе\thispace\thinspaceл\thispace\thinspaceь\thispace\thinspaceс\thispace\thinspaceт\thispace\thinspaceв\thispace\thinspaceо.}
$f^{(n-1)}(x)$ существует при $|x-\xi|<\epsilon$ для некоторо $\epsilon>0$. Тогда из теорем $177$ и $178$ ( с заменой $n$ на $n-1$) следует, что при $0<|h|<\epsilon$ между $\xi$ и $\xi+h$ имеется $y$, такое, что:
\begin{equation}\label{formula1}
f(\xi+h)-f(\xi)=\frac{h^{n-1}}{(n-1)!}f^{(n-1)}(y).
\end{equation}
$f^{n-1}(x)$ возрастает, соответственно убывает, в $\xi$, смотря по тому, будет ли $f^n(\xi)>0$, соответственно $<0$. Следовательно, существует $\epsilon_1, 0<\epsilon_1<\epsilon$, такое, что при $0<|h|<\epsilon_1$ для всех $y$ между $\xi$ и $\xi+h$ выполняется неравенство
\begin{equation}\label{formula2}
hf^{n-1}(y)f^n(\xi)>0.
\end{equation}
Поэтому из (1) имеем при $0<|h|<\xi_1$, с тамошним y
\begin{center}$h^{n}f^{(n)}(f(\xi+h)-f(\xi))=(\frac{(h''-1)^2}{(n-1)!}hf^{(n-1)}(y)f^{(n)}(\xi)>0$.\end{center}
Следовательно,
\par1) Если $n$ четное и $f^{(n)}(\xi)>0$, то
\begin{center}$f(\xi+h)-f(\xi)>0$.\end{center}
\par2) Если $n$ четное и $f^{(n)}(\xi)<0$, то
\begin{center}$f(\xi+h)-f(\xi)<0$.\end{center}
\par3) Если $n$ нечетное и $f^{(n)}(\xi)>0$, то
\begin{center}$h(f(\xi+h)-f(\xi))>0$.\end{center}
\par4) Если $n$ нечетное и $f^{(n)}(\xi)<0$, то
\begin{center}$h(f(\xi+h)-f(\xi))>0$.\end{center}
Примеры \RomanNumeralCaps{1})-\RomanNumeralCaps{4}) --- те же, что и в конце гл. 7.\\
\RomanNumeralCaps{1}) $f(x)=-x^2$, $f'(0)=0$, $f''(0)=-2<0$: максимум в 0.\\
\RomanNumeralCaps{2}) $f(x)=-x^2$, $f'(0)=0$, $f''(0)=-2<0$: минимум в 0.\\
\RomanNumeralCaps{3}) $f(x)=-x^3$, $f'(0)=0$, $f''(0)=0$, $f'''(0)=6>0$: возрастание в 0.\\
\newpage