\footnotesize
\fancyhead[C]{\itПроизводные высших порядков}
\fancyhead[R]{\it151}
\linespread{1}
\normalsize
\RomanNumeralCaps{4}) $f(x)=-x^3$, $f'(0)=0$, $f''(0)=0$, $f'''(0)=-6<0$: убывание в 0.
\par\RomanNumeralCaps{5}) Пусть
\begin{center}
$f(x)=\frac{(x+1)^3}{x^2}$ при $x\neq0$.
\end{center}
Найдём все максимумы и минимумы этой функции. Так как, очевидно, $f(x)$ при $x\neq0$ дифференцируема любое число раз (поскольку этим свойством обладают $x^{-2}$ и $(x+3)^3$), то подозрительными на максимум или минимум являются лишь корни функции $f'(x)$ (т. е. те $x$, для которых $f'(x)=0$). Но при $x\neq0$ имеем
\begin{center}
$f'(x)=(x^{-2}(x+1)^3)'=-2x^{-3}(x+1)^3+3x^{-2}(x+1)^2=$\\
$=x^{-3}(x+1)^2(-2x-2+3x)=x^{-3}(x+1)^2(x-2)$.
\end{center}
Таким образом, исследованию подлежат лишь $x=-1$ и $x=2$. Для этого мы применим наш признак, не вычисляя, однако, ненужных членов.
\parИсследование $x=-1$:
\par\par$f''(x)=\Big((x+1)^2\frac{x-2}{x^3}\Big)'=$
\begin{center}
$=(x+1)^2\Big(\frac{x-2}{x^3}\Big)'+(x+1)\frac{x-2}{x^3}$,\\
$f''(-1)=0$,\\
$f'''(x)=\Big((x+1)^2\frac{x-2}{x^3}\Big)''=$\\
$=(x+1)^2\Big(\frac{x-2}{x^3}\Big)''+4(x+1)\Big(\frac{x-2}{x^3}\Big)'+2\frac{x-2}{x^3}$\\
$f'''(-1)=0+0+2\frac{-3}{-1}>0$.
\end{center}
Возрастание, ни максимума, ни минимума.
\parИсследование $x=2$:
\begin{center}
$f''(2)=\lim\limits_{x= 2} \frac{f'(x)}{x-2}=\lim\limits_{x= 2} \frac{(x+1)^2}{x^3}>0$.
\end{center}
Минимум,